\documentclass[a4paper, 11pt]{article}
\usepackage[utf8]{inputenc}
\usepackage[T1]{fontenc}
\usepackage{comment} % enables the use of multi-line comments (\ifx \fi) 
\usepackage{lipsum} % generates Lorem Ipsum filler text
\usepackage[colorlinks=true,linkcolor=blue,urlcolor=black,bookmarksopen=true]{hyperref}
\usepackage{bookmark}
\usepackage{color}
\usepackage{marvosym}
\usepackage[ampersand]{easylist}
\usepackage{fancyhdr}
\usepackage{lastpage}
\usepackage[margin=1in,headheight=13.6pt]{geometry}
\usepackage{graphicx}
\usepackage[whole]{bxcjkjatype}
\usepackage{longtable}
\usepackage{booktabs}
\usepackage{tikz}
\input{chronos.txt}

\newcommand{\ra}[1]{\renewcommand{\arraystretch}{#1}}

\setlength{\parindent}{0em}
\setlength{\parskip}{1em}

\pagestyle{fancy}
\fancyhf{}
\chead{\small \textless{}Project Name\textgreater{}}
\renewcommand{\footrulewidth}{0.5pt}
\cfoot{\small Page \thepage\ of \pageref{LastPage}}

%----------------------------------------------------------

\begin{document}
\noindent

\vspace*{240pt}
\hfill\textbf{\Large \textless{}Project Name\textgreater{}}

\hfill\textbf{\Large PROJECT CHARTER}

\vspace{5pt}
\hrule

\hfill Version: 1.0

\hfill Date: \today
  
\newpage

%----------------------------------------------------------

\textbf{VERSION HISTORY}

{[}Provide information on how the development and distribution of the
Project Charter up to the final point of approval was controlled and
tracked. Use the table below to provide the version number, the author
implementing the version, the date of the version, the name of the
person approving the version, the date that particular version was
approved, and a brief description of the reason for creating the revised
version.{]}

\ra{1.3}
\begin{longtable}{@{}llllll@{}}
  \toprule
  \begin{minipage}[t]{0.14\columnwidth}\raggedright
  \textbf{Version\\\#}\strut
  \end{minipage} & \begin{minipage}[t]{0.14\columnwidth}\raggedright
  \textbf{Implemented\\By}\strut
  \end{minipage} & \begin{minipage}[t]{0.14\columnwidth}\raggedright
  \textbf{Revision\\Date}\strut
  \end{minipage} & \begin{minipage}[t]{0.14\columnwidth}\raggedright
  \textbf{Approved\\By}\strut
  \end{minipage} & \begin{minipage}[t]{0.14\columnwidth}\raggedright
  \textbf{Approval\\Date}\strut
  \end{minipage} & \begin{minipage}[t]{0.14\columnwidth}\raggedright
  \textbf{Reason}\strut
  \end{minipage}\tabularnewline      
  \midrule
  \endhead
  1.0 & \emph{\textless{}Author name\textgreater{}} &
  \emph{\textless{}mm/dd/yy\textgreater{}} &
  \emph{\textless{}name\textgreater{}} &
  \emph{\textless{}mm/dd/yy\textgreater{}} &
  \emph{\textless{}reason\textgreater{}} \tabularnewline
  1.1 & \emph{\textless{}Author name\textgreater{}} &
  \emph{\textless{}mm/dd/yy\textgreater{}} &
  \emph{\textless{}name\textgreater{}} &
  \emph{\textless{}mm/dd/yy\textgreater{}} &
  \emph{\textless{}reason\textgreater{}} \tabularnewline
  & & & & & \tabularnewline
  & & & & & \tabularnewline
  & & & & & \tabularnewline
  \bottomrule
\end{longtable}

\hfill \textbf{UP Template Version:} 11/30/06

\newpage

%----------------------------------------------------------

\textbf{Note to the author}

{[}This document is a template of a Project Charter document for a
project. The template includes instructions to the author, boilerplate
text, and fields that should be replaced with the values specific to the
project.

\begin{itemize}
\item
  Blue italicized text enclosed in square brackets ({[}text{]}) provides
  instructions to the document author, or describes the intent,
  assumptions and context for content included in this document.
\item
  Blue italicized text enclosed in angle brackets
  (\textless{}text\textgreater{}) indicates a field that should be
  replaced with information specific to a particular project.
\item
  Text and tables in black are provided as boilerplate examples of
  wording and formats that may be used or modified as appropriate to a
  specific project. These are offered only as suggestions to assist in
  developing project documents; they are not mandatory formats.
\end{itemize}

When using this template for your project document, it is recommended
that you follow these steps:

\begin{enumerate}
\def\labelenumi{\arabic{enumi}.}
\item
  Replace all text enclosed in angle brackets (i.e., \textless{}Project
  Name\textgreater{}) with the correct field values. These angle
  brackets appear in both the body of the document and in headers and
  footers. To customize fields in Microsoft Word (which display a gray
  background when selected):

  \begin{enumerate}
  \def\labelenumii{\alph{enumii}.}
  \item
    Select File\textgreater{}Properties\textgreater{}Summary and fill in
    the Title field with the Document Name and the Subject field with
    the Project Name.
  \item
    Select File\textgreater{}Properties\textgreater{}Custom and fill in
    the Last Modified, Status, and Version fields with the appropriate
    information for this document.
  \item
    After you click OK to close the dialog box, update the fields
    throughout the document with these values by selecting
    Edit\textgreater{}Select All (or Ctrl-A) and pressing F9. Or you can
    update an individual field by clicking on it and pressing F9. This
    must be done separately for Headers and Footers.
  \end{enumerate}
\end{enumerate}

\begin{enumerate}
\def\labelenumi{\arabic{enumi}.}
\item
  Modify boilerplate text as appropriate to the specific project.
\item
  To add any new sections to the document, ensure that the appropriate
  header and body text styles are maintained. Styles used for the
  Section Headings are Heading 1, Heading 2 and Heading 3. Style used
  for boilerplate text is Body Text.
\item
  To update the Table of Contents, right-click and select ``Update
  field'' and choose the option- ``Update entire table''
\item
  Before submission of the first draft of this document, delete this
  ``Notes to the Author'' page and all instructions to the author, which
  appear throughout the document as blue italicized text enclosed in
  square brackets.{]}
\end{enumerate}

\newpage

%----------------------------------------------------------

\tableofcontents

\newpage

%----------------------------------------------------------

\hypertarget{introduction}{%
\section{Introduction}\label{introduction}}

\hypertarget{purpose-of-project-charter}{%
\subsection{Purpose of Project
Charter}\label{purpose-of-project-charter}}

{[}Provide the purpose of the project charter.{]}

The \emph{\textless{}Project Name\textgreater{}} project charter
documents and tracks the necessary information required by decision
maker(s) to approve the project for funding. The project charter should
include the needs, scope, justification, and resource commitment as well
as the project's sponsor(s) decision to proceed or not to proceed with
the project. It is created during the Initiating Phase of the project.

The intended audience of the \emph{\textless{}Project
Name\textgreater{}} project charter is \emph{the project sponsor and
senior leadership.}

\hypertarget{project-and-product-overview}{%
\section{Project And Product Overview}\label{project-and-product-overview}}

{[}Typically, the description should answer who, what, when and where,
in a concise manner. It should also state the estimated project duration
(e.g., 18 months) and the estimated project budget (e.g., \$1.5M).

\hypertarget{justification}{%
\section{Justification}\label{justification}}

\hypertarget{business-need}{%
\subsection{Business Need}\label{business-need}}

{[}Example: A data collection system is necessary to conduct a national
program of surveillance and research to monitor and characterize the x
epidemic, including its determinants and the epidemiologic dynamics such
as prevalence, incidence, and antiretroviral resistance, and to guide
public health action at the federal, state and local levels. Data
collection activities will assist with monitoring the incidence and
prevalence of x infection, and x-related morbidity and mortality in the
population, estimate incidence of x infection, identify changes in
trends of x transmission, and identify populations at risk.){]}

\hypertarget{public-health-and-business-impact}{%
\subsection{Public Health and Business
Impact}\label{public-health-and-business-impact}}

{[}Example: System \emph{x} collects information about \emph{x}
infection as the jurisdictional, regional, and national levels and will
assist in monitoring trends in x transmission rates, incidence rates and
x morbidity and mortality trends to help determine public health
impact.{]}

\hypertarget{strategic-alignment}{%
\subsection{Strategic Alignment}\label{strategic-alignment}}

\lipsum[1]

\ra{1.3}
\begin{longtable}[]{@{}l|l|l@{}}
  \toprule
  \textbf{Goal} & \textbf{Project Response Rank} &
  \textbf{Comments}\tabularnewline
  \midrule
  \multicolumn{3}{c}{
  \emph{Scale}: \textbf{H} -- High, \textbf{M} -- Medium, \textbf{L} -- Low,
  \textbf{N/A} -- Not Applicable }\tabularnewline
  \midrule
  \endhead
  \multicolumn{3}{@{}l}{
  \textbf{NC / Division / Branch Strategic Goals:} }\tabularnewline
  \midrule
  combo & & \tabularnewline
  & & \tabularnewline
  \midrule
  \multicolumn{3}{@{}l}{
  \textbf{CDC Strategic Goals:} }\tabularnewline
  \midrule
  \emph{\textless{}Reference Appendix C for goals\textgreater{}} & & \tabularnewline
  & & \tabularnewline
  \midrule
  \multicolumn{3}{@{}l}{
  \textbf{Department of Health and Human Services (DHHS) Strategic Goals:} }\tabularnewline
  \midrule
  \emph{\textless{}Reference Appendix C for goals\textgreater{}} & & \tabularnewline
  & & \tabularnewline
  \midrule
  \multicolumn{3}{@{}l}{
  \textbf{DHHS IT Goals:} }\tabularnewline
  \midrule
  \emph{\textless{}Reference Appendix C for goals\textgreater{}} & & \tabularnewline
  & & \tabularnewline
  \midrule
  \multicolumn{3}{@{}l}{
  \textbf{President's Management Agenda (PMA) Strategic Goals:} }\tabularnewline
  \midrule
  \emph{\textless{}Reference Appendix C for goals\textgreater{}} & & \tabularnewline
  & & \tabularnewline
  \bottomrule
\end{longtable}

\hypertarget{scope}{%
\section{Scope}\label{scope}}

\hypertarget{objectives}{%
\subsection{Objectives}\label{objectives}}

{[}Example: Improving epidemiologic analyses by provisioning consistent
data or to making progress towards a 2010 goal{]}

The objectives of the \emph{\textless{}Project Name\textgreater{}} are
as follows:

\begin{itemize}
\item
  \emph{{[}Insert Objective 1{]}}
\item
  \emph{{[}Insert Objective 2{]}}
\item
  \emph{{[}Add additional bullets as necessary{]}}
\end{itemize}

\hypertarget{high-level-requirements}{%
\subsection{High-Level Requirements}\label{high-level-requirements}}

The following table presents the requirements that the project's
product, service or result must meet in order for the project objectives
to be satisfied.

\ra{1.3}
\begin{longtable}[]{@{}ll@{}}
  \toprule
  \textbf{Req. \#} & \textbf{Requirement Description}\tabularnewline
  \midrule
  \endhead
  &\tabularnewline
  &\tabularnewline
  \bottomrule
\end{longtable}

\hypertarget{major-deliverables}{%
\subsection{Major Deliverables}\label{major-deliverables}}

The following table presents the major deliverables that the project's
product, service or result must meet in order for the project objectives
to be satisfied.

\ra{1.3}
\begin{longtable}[]{@{}ll@{}}
  \toprule
  \textbf{Major Deliverable} & \textbf{Deliverable Description}\tabularnewline
  \midrule
  \endhead
  &\tabularnewline
  &\tabularnewline
  \bottomrule
\end{longtable}

\hypertarget{boundaries}{%
\subsection{Boundaries}\label{boundaries}}

{[}Describe the inclusive and exclusive boundaries of the project.
Specifically address items that are out of scope.{]}

\hypertarget{duration}{%
\section{Duration}\label{duration}}

\hypertarget{timeline}{%
\subsection{Timeline}\label{timeline}}

{[}An example of a high-level timeline is provided below.{]}

%---------------------------------------
\begin{tikzpicture}
  [
    chronos={%
      start date={0-01-01},
      end date={2000-01-01},
      step years=250,
      timeline height=1pt,
      timeline width=\textwidth,
      timeline line={shorten >={-10mm}, -{Triangle Cap[length=10mm]}},
      date format={!Y/!m/!d},
    }
  ]
  \chronosperiod[draw=gray]{476-01-01}{476-10-31}{Fall of the\\Roman Empire}
  \chronosevent{1492-10-11}{European\\Re-Discovery\\of America}
  \chronosevent{1969-07-21}{First Steps\\on the Moon}
\end{tikzpicture}

Problem: we must include two Jan-01 days otherwise latex will hang up.

\begin{tikzpicture}
  [
    chronos={%
      start date={2019-01-01},
      end date={2020-06-01},
      step years=250,
      timeline height=1pt,
      timeline width=\textwidth,
      timeline line={shorten >={-10mm}, -{Triangle Cap[length=10mm]}},
      date format={!m/!d},
    }
  ]
  \chronosevent{2019-10-05}{Project Plan\\Completed}
  \chronosevent{2019-12-05}{Requrements\\Analysis\\Completed}
  \chronosevent{2020-02-06}{Developed\\Prototype}
  \chronosevent{2020-05-06}{System\\Development\\Completed}
\end{tikzpicture}
%---------------------------------------

\hypertarget{executive-milestones}{%
\subsection{Executive Milestones }\label{executive-milestones}}

{[}Example: For CPIC major/tactical projects, these milestones could be
used to complete the Funding Plan/Cost and Schedule section of the OMB
Exhibit 300.{]}

The table below lists the high-level Executive Milestones of the project
and their estimated completion timeframe.

\ra{1.3}
\begin{longtable}[]{@{}ll@{}}
  \toprule
  \textbf{Executive Milestones} & \textbf{Estimated Completion Timeframe}\tabularnewline
  \midrule
  \endhead
  {[}Insert milestone information (e.g., Project planned and authorized to
  proceed){]} & {[}Insert completion timeframe (e.g., Two weeks after
  project concept is approved){]}\tabularnewline
  {[}Insert milestone information (e.g., Version 1 completed){]} &
  {[}Insert completion timeframe (e.g., Twenty-five weeks after
  requirements analysis is completed){]}\tabularnewline
  {[}Add additional rows as necessary{]} &\tabularnewline
  \bottomrule
\end{longtable}

\hypertarget{budget-estimate}{%
\section{Budget Estimate}\label{budget-estimate}}

\hypertarget{funding-source}{%
\subsection{Funding Source}\label{funding-source}}

{[}Example: grant, terrorism budget, or operational budget.{]}

\hypertarget{estimate}{%
\subsection{Estimate}\label{estimate}}

This section provides a summary of estimated spending to meet the
objectives of the \emph{\textless{}Project Name\textgreater{}} project
as described in this project charter. This summary of spending is
preliminary, and should reflect costs for the entire investment
lifecycle. It is intended to present probable funding requirements and
to assist in obtaining budgeting support.\\
\emph{{[}For CPIC major/tactical projects complete and attach the
required sections of the \href{http://intranet.cdc.gov/cpic/}{OMB
Exhibit 300} located at \url{http://intranet.cdc.gov/cpic/}. For all
other projects, provide a summary of the project's expected spending}
\emph{below.{]}}

\textbf{MISSING TABLE} %==========

\hypertarget{high-level-alternatives-analysis}{%
\section{High-Level Alternatives Analysis}\label{high-level-alternatives-analysis}}

{[}Example: Alternatives to developing a custom system may have included
looking at existing COTS products or reusing an existing system.{]}

\begin{enumerate}
\def\labelenumi{\arabic{enumi}.}
\item
  \emph{{[}Provide a statement summarizing the factors considered.{]}}
\item
  \emph{{[}Provide a statement summarizing the factors considered.{]}}
\end{enumerate}

\hypertarget{assumptions-constraints-and-risks}{%
\section{Assumptions, Constraints And Risks}\label{assumptions-constraints-and-risks}}

\hypertarget{assumptions}{%
\subsection{Assumptions}\label{assumptions}}

{[}Example: The system is being developed to capture data from public
health partners. One assumption is that data is entered electronically
into the system.{]}

This section identifies the statements believed to be true and from
which a conclusion was drawn to define this project charter.

\begin{enumerate}
\def\labelenumi{\arabic{enumi}.}
\item
  \begin{quote}
  {[}Insert description of the first assumption.{]}
  \end{quote}
\item
  \begin{quote}
  \emph{{[}Insert description of the second assumption.{]}}
  \end{quote}
\end{enumerate}

\hypertarget{constraints}{%
\subsection{Constraints}\label{constraints}}

{[}Example: There might be time constraints on developing a system that
is used to track data of highly infectious diseases like SARS.{]}

This section identifies any limitation that must be taken into
consideration prior to the initiation of the project.

\begin{enumerate}
\def\labelenumi{\arabic{enumi}.}
\item
  \emph{{[}Insert description of the first constraint.{]}}
\item
  \emph{{[}Insert description of the second constraint.{]}}
\end{enumerate}

\hypertarget{risks}{%
\subsection{Risks}\label{risks}}

{[}Example: The risk of accessibility or unavailability of public health
partners for obtaining requirements to develop a data collection system
may delay project deliverables. A possible mitigation strategy might be
to schedule requirement sessions with the partners as early as possible.
List the risks that the project sponsor should be aware of before making
a decision on funding the project, including risks of not funding the
project.{]}

\ra{1.3}
\begin{longtable}[]{@{}ll@{}}
  \toprule
  \textbf{Risk} & \textbf{Mitigation}\tabularnewline
  \midrule
  \endhead
  &\tabularnewline
  &\tabularnewline
  \bottomrule
\end{longtable}

\hypertarget{project-organization}{%
\section{Project Organization}\label{project-organization}}

\hypertarget{roles-and-responsibilities}{%
\subsection{Roles and
Responsibilities}\label{roles-and-responsibilities}}

{[}Depending on your project organization, you may modify the roles and
responsibilities listed in the table below.{]}

This section describes the key roles supporting the project.

\ra{1.3}
\begin{longtable}[]{@{}lll@{}}
  \toprule
  \textbf{Name \& Organization} & \textbf{Project Role} & \textbf{Project
  Responsibilities}\tabularnewline
  \midrule
  \endhead
  \begin{minipage}[t]{0.30\columnwidth}\raggedright
  \textless{}Name\textgreater{}

  \textless{}Org\textgreater{}\strut
  \end{minipage} & \begin{minipage}[t]{0.30\columnwidth}\raggedright
  Project Sponsor\strut
  \end{minipage} & \begin{minipage}[t]{0.30\columnwidth}\raggedright
  Person responsible for acting as the project's champion and providing
  direction and support to the team. In the context of this document, this
  person approves the request for funding, approves the project scope
  represented in this document, and sets the priority of the project
  relative to other projects in his/her area of responsibility.\strut
  \end{minipage}\tabularnewline
  \begin{minipage}[t]{0.30\columnwidth}\raggedright
  \textless{}Name\textgreater{}

  \textless{}Org\textgreater{}\strut
  \end{minipage} & \begin{minipage}[t]{0.30\columnwidth}\raggedright
  Government Monitor\strut
  \end{minipage} & \begin{minipage}[t]{0.30\columnwidth}\raggedright
  Government employee who provides the interface between the project team
  and the project sponsor. Additionally, they will serve as the single
  focal point of contact for the Project Manager to manage CDC's
  day-to-day interests. This person must have adequate business and
  project knowledge in order to make informed decisions.

  In the case where a contract is involved, the role of a Government
  Monitor will often be fulfilled by a Contracting Officer and a Project
  Officer.\strut
  \end{minipage}\tabularnewline
  \begin{minipage}[t]{0.30\columnwidth}\raggedright
  \textless{}Name\textgreater{}

  \textless{}Org\textgreater{}\strut
  \end{minipage} & \begin{minipage}[t]{0.30\columnwidth}\raggedright
  Contracting Officer\strut
  \end{minipage} & \begin{minipage}[t]{0.30\columnwidth}\raggedright
  Person who has the authority to enter into, terminate, or change a
  contractual agreement on behalf of the Government. This person bears the
  legal responsibility for the contract.\strut
  \end{minipage}\tabularnewline
  \begin{minipage}[t]{0.30\columnwidth}\raggedright
  \textless{}Name\textgreater{}

  \textless{}Org\textgreater{}\strut
  \end{minipage} & \begin{minipage}[t]{0.30\columnwidth}\raggedright
  Project Officer\strut
  \end{minipage} & \begin{minipage}[t]{0.30\columnwidth}\raggedright
  A program representative responsible for coordinating with acquisition
  officials on projects for which contract support is contemplated. This
  representative is responsible for technical monitoring and evaluation of
  the contractor's performance after award.\strut
  \end{minipage}\tabularnewline
  \begin{minipage}[t]{0.30\columnwidth}\raggedright
  \textless{}Name\textgreater{}

  \textless{}Org\textgreater{}\strut
  \end{minipage} & \begin{minipage}[t]{0.30\columnwidth}\raggedright
  Project Manager (This could include a Contractor Project Manager or an
  FTE Project Manager)\strut
  \end{minipage} & \begin{minipage}[t]{0.30\columnwidth}\raggedright
  Person who performs the day-to-day management of the project and has
  specific accountability for managing the project within the approved
  constraints of scope, quality, time and cost, to deliver the specified
  requirements, deliverables and customer satisfaction.\strut
  \end{minipage}\tabularnewline
  \begin{minipage}[t]{0.30\columnwidth}\raggedright
  \textless{}Name\textgreater{}

  \textless{}Org\textgreater{}\strut
  \end{minipage} & \begin{minipage}[t]{0.30\columnwidth}\raggedright
  Business Steward\strut
  \end{minipage} & \begin{minipage}[t]{0.30\columnwidth}\raggedright
  Person in management, often the Branch Chief or Division Director, who
  is responsible for the project in its entirety.\strut
  \end{minipage}\tabularnewline
  \begin{minipage}[t]{0.30\columnwidth}\raggedright
  \textless{}Name\textgreater{}

  \textless{}Org\textgreater{}\strut
  \end{minipage} & \begin{minipage}[t]{0.30\columnwidth}\raggedright
  Technical Steward\strut
  \end{minipage} & \begin{minipage}[t]{0.30\columnwidth}\raggedright
  Person who is responsible for the technical day-to-day aspects of the
  system including the details of system development. The Technical
  Steward is responsible for providing technical direction to the
  project.\strut
  \end{minipage}\tabularnewline
  \begin{minipage}[t]{0.30\columnwidth}\raggedright
  \textless{}Name\textgreater{}

  \textless{}Org\textgreater{}\strut
  \end{minipage} & \begin{minipage}[t]{0.30\columnwidth}\raggedright
  Security Steward\strut
  \end{minipage} & \begin{minipage}[t]{0.30\columnwidth}\raggedright
  Person who is responsible for playing the lead role for maintaining the
  project's information security.\strut
  \end{minipage}\tabularnewline
  \bottomrule
\end{longtable}

\hypertarget{stakeholders-internal-and-external}{%
\subsection{Stakeholders (Internal and
External)}\label{stakeholders-internal-and-external}}

{[}Examples of stakeholders include an epidemiologist performing a
behavioral research project and people in the field collecting data
using a software application (the proposed project) to collect the data
required for a behavioral research project.{]}

\newpage

%----------------------------------------------------------

\hypertarget{project-charter-approval}{%
\section{Project Charter Approval}\label{project-charter-approval}}

The undersigned acknowledge they have reviewed the project charter and
authorize and fund the \emph{\textless{}Project Name\textgreater{}}
project. Changes to this project charter will be coordinated with and
approved by the undersigned or their designated representatives.

{[}List the individuals whose signatures are desired. Examples of such
individuals are Business Steward, Project Manager or Project Sponsor.
Add additional lines for signature as necessary. Although signatures are
desired, they are not always required to move forward with the practices
outlined within this document.{]}

\ra{2.0}
\begin{longtable}[]{@{}llll@{}}
  Signature: & \hspace{100pt} & Date: & \hspace*{100pt}\tabularnewline
  Print Name: & & &\tabularnewline
  Title: & & &\tabularnewline
  Role: & & &\tabularnewline
\end{longtable}

\begin{longtable}[]{@{}llll@{}}
  Signature: & \hspace{100pt} & Date: & \hspace*{100pt}\tabularnewline
  Print Name: & & &\tabularnewline
  Title: & & &\tabularnewline
  Role: & & &\tabularnewline
\end{longtable}

\begin{longtable}[]{@{}llll@{}}
  Signature: & \hspace{100pt} & Date: & \hspace*{100pt}\tabularnewline
  Print Name: & & &\tabularnewline
  Title: & & &\tabularnewline
  Role: & & &\tabularnewline
\end{longtable}

\newpage

%----------------------------------------------------------

\section{APPENDIX A: REFERENCES}

{[}Insert the name, version number, description, and physical location
of any documents referenced in this document. Add rows to the table as
necessary.{]}

\begin{quote}
The following table summarizes the documents referenced in this
document.
\end{quote}

\ra{1.3}
\begin{longtable}[]{@{}lll@{}}
  \toprule
  \textbf{Document Name and Version} & \textbf{Description} &
  \textbf{Location}\tabularnewline
  \midrule
  \endhead
  \emph{\textless{}Document Name and Version Number\textgreater{}} &
  \emph{{[}Provide description of the document{]}} & \emph{\textless{}URL
  or Network path where document is located\textgreater{}}\tabularnewline
  \bottomrule
\end{longtable}

\newpage

%----------------------------------------------------------

\section{APPENDIX B: KEY TERMS}

\emph{{[}Insert terms and definitions used in this document. Add rows to
the table as necessary. Follow the link below to for definitions of
project management terms and acronyms used in this and other documents.}

\emph{http://www2.cdc.gov/cdcup/library/other/help.htm}

The following table provides definitions for terms relevant to this
document.

\ra{1.3}
\begin{longtable}[]{@{}ll@{}}
  \toprule
  \textbf{Term} & \textbf{Definition}\tabularnewline
  \midrule
  \endhead
  \emph{{[}Insert Term{]}} & \emph{{[}Provide definition of the term used
  in this document.{]}}\tabularnewline
  \emph{{[}Insert Term{]}} & \emph{{[}Provide definition of the term used
  in this document.{]}}\tabularnewline
  \emph{{[}Insert Term{]}} & \emph{{[}Provide definition of the term used
  in this document.{]}}\tabularnewline
  \bottomrule
\end{longtable}

\newpage

%----------------------------------------------------------

\section{APPENDIX C: GOALS}

\begin{itemize}
\item
  \begin{quote}
  \textbf{CDC Strategic Goals}\\
  \textbf{URL}:
  \href{http://www.cdc.gov/about/goals/}{\textbf{http://www.cdc.gov/about/goals/}}
  \end{quote}

  \begin{itemize}
  \item
    \begin{quote}
    \textbf{Goal 1} - Healthy People in Every Stage of Life
    \end{quote}
  \item
    \begin{quote}
    \textbf{Goal 2} - Healthy People in Healthy Places
    \end{quote}
  \item
    \begin{quote}
    \textbf{Goal 3} - People Prepared for Emerging Health Threats
    \end{quote}
  \item
    \begin{quote}
    \textbf{Goal 4} - Healthy People in a Healthy World
    \end{quote}
  \end{itemize}
\item
  \begin{quote}
  \textbf{Department of Health and Human Services (DHHS) Strategic Goals}\\
  \textbf{URL}:
  \href{http://aspe.hhs.gov/hhsplan/2004/goals.shtml}{\textbf{http://aspe.hhs.gov/hhsplan/2004/goals.shtml}}
  (Search for ``HHS IT Strategic Plan'')
  \end{quote}

  \begin{itemize}
  \item
    \begin{quote}
    \textbf{Goal 1} - Reduce the major threats to the health and
    well-being of Americans
    \end{quote}
  \item
    \begin{quote}
    \textbf{Goal 2} - Enhance the ability of the Nation's health care
    system to effectively respond to bioterrorism and other public
    health challenges
    \end{quote}
  \item
    \begin{quote}
    \textbf{Goal 3} - Increase the percentage of the Nation's children
    and adults who have access to health care services, and expand
    consumer choices
    \end{quote}
  \item
    \begin{quote}
    \textbf{Goal 4} - Enhance the capacity and productivity of the
    Nation's health science research enterprise
    \end{quote}
  \item
    \begin{quote}
    \textbf{Goal 5} - Improve the quality of health care services
    \end{quote}
  \item
    \begin{quote}
    \textbf{Goal 6} - Improve the economic and social well-being of
    individuals, families, and communities, especially those most in
    need
    \end{quote}
  \item
    \begin{quote}
    \textbf{Goal 7} - Improve the stability and healthy development of
    our Nation's children and youth
    \end{quote}
  \item
    \begin{quote}
    \textbf{Goal 8} - Achieve excellence in management practices
    \end{quote}
  \end{itemize}
\item
  \begin{quote}
  \textbf{Department of Health and Human Services (DHHS) IT Goals}\\
  \textbf{URL}:
  \href{http://aspe.hhs.gov/hhsplan/2004/goals.shtml}{\textbf{http://aspe.hhs.gov/hhsplan/2004/goals.shtml}}
  \end{quote}

\begin{itemize}
\item
  \textbf{Goal 1} - Provide a secure and trusted IT environment
\item
  \textbf{Goal 2} - Enhance the quality, availability, and delivery of
  HHS information and services to citizens, employees, businesses, and
  governments
\item
  \textbf{Goal 3} - Implement an enterprise approach to IT
  infrastructure and common administrative systems that will foster
  innovation and collaboration
\item
  \textbf{Goal 4} - Enable and improve the integration of health and
  human services information
\item
  \textbf{Goal 5} - Achieve excellence in IT management practices,
  including a governance process that complements program management,
  supports e-government initiatives, and ensures effective data privacy
  and information security controls
\end{itemize}

\item
  \begin{quote}
  \textbf{President's Management Agenda (PMA) Strategic Goals}\\
  \textbf{URL}:
  \href{http://www.whitehouse.gov/omb/budintegration/pma_index.html}{\textbf{http://www.whitehouse.gov/omb/budintegration/pma\_index.html}}
  \end{quote}

\begin{quote}
\textbf{Government-wide Initiatives}
\end{quote}

\begin{itemize}
\item
  \begin{quote}
  \textbf{Goal 1} - Strategic Management of Human Capital
  \end{quote}
\item
  \begin{quote}
  \textbf{Goal 2} - Competitive Sourcing
  \end{quote}
\item
  \begin{quote}
  \textbf{Goal 3} - Improved Financial Performance
  \end{quote}
\item
  \begin{quote}
  \textbf{Goal 4} - Expanded Electronic Government
  \end{quote}
\item
  \begin{quote}
  \textbf{Goal 5} - Budget and Performance Integration
  \end{quote}
\end{itemize}

\begin{quote}
\textbf{Program Initiatives}
\end{quote}

\begin{itemize}
\item
  \begin{quote}
  \textbf{Goal 1} - Faith-Based and Community Initiative
  \end{quote}
\item
  \begin{quote}
  \textbf{Goal 2} - Privatization of Military Housing
  \end{quote}
\item
  \begin{quote}
  \textbf{Goal 3} - Better Research and Development Investment Criteria
  \end{quote}
\item
  \begin{quote}
  \textbf{Goal 4} - Elimination of Fraud and Error in Student Aid
  Programs and Deficiencies in Financial Management
  \end{quote}
\item
  \begin{quote}
  \textbf{Goal 5} - Housing and Urban Development Management and
  Performance
  \end{quote}
\item
  \begin{quote}
  \textbf{Goal 6} - Broadened Health Insurance Coverage through State
  Initiatives
  \end{quote}
\item
  \begin{quote}
  \textbf{Goal 7} - A ``Right-Sized'' Overseas Presence
  \end{quote}
\item
  \begin{quote}
  \textbf{Goal 8} - Reform of Food Aid Programs
  \end{quote}
\item
  \begin{quote}
  \textbf{Goal 9} - Coordination of Veterans Affairs and Defense
  Programs and Systems
  \end{quote}
\end{itemize}
\end{itemize}

%----------------------------------------------------------
\end{document}
